\documentclass[11pt]{report}
%\usepackage{fancybox}
\usepackage{geometry}
\usepackage{amsmath}
\usepackage{txfonts}
\usepackage{layout}
\usepackage{setspace}
%\usepackage{mathptmx}
\geometry{a4paper, left=22mm, right=22mm, top=25mm, bottom=25mm}
\usepackage{wrapfig}
\usepackage[dvipdfm]{graphicx,hyperref}
\usepackage{mediabb}
\setstretch{1.3}
\title{\Huge\bf{Role of Noncollective Excitations in Low-Energy Heavy-Ion Fusion Reaction
and Quasi-Elastic Scattering}}
\author{\\\\\\\\\\\\\\\\\\\\\\\\\\\\{\it\Large Department of Physics, Faculty of Science, Tohoku University}\\\\
\Huge Shusaku Yusa}
\date{March, 2013}
\begin{document}


\chapter{Summary and concluding remarks}
We have investigated the role of noncollective excitations in heavy-ion
fusion reaction and quasi-elastic scattering by explicitly taking into account
the noncollective states in coupled-channels calculations.
In chapter 5, we considered $^{16}$O + $^{208}$Pb system
because this system has been
studied both from the experimental and theoretical sides,
and the experimental Q-value distribution suggests the contribution
from the noncollective excitations.
In addition, almost all the
excited states of $^{208}$Pb up to about 7 MeV have been identified by high
precision proton inelastic scattering experiments.
We used the experimentally obtained excitation energy, multipolarity, and
deformation parameter of the excited states in $^{208}$Pb to describe the
noncollective excitations in $^{16}$O + $^{208}$Pb reaction.

Our results show that the barrier distribution for the fusion reaction and the
quasi-elastic scattering are changed in a similar manner due to the
noncollective excitations at energies above the Coulomb barrier. The energy
dependence of the cross sections, on the other hand, is not affected much by the
noncollective excitations and the degree of agreement with the experimental
barrier distribution remains the same.
The effect of anharmonicity of vibrational states in $^{208}$Pb is also investigated.
It has been found that the effect of anharmonicity plays a minor role regardless
of the presence of the noncollective excitations.
% We have, therefore, ruled out coupling to
% the many noncollective excitations as a possible source of the hindrance of
% fusion cross sections at deep subbarrier energies and at energies above the
% Coulomb barrier. However, this important result still leaves the mechanism for
% fusion hindrance phenomena as an open question. In order to improve the
% agreement, one would have to consider another mechanism, such as a reduction of
% the excitation energy of the 3$^-$ state in $^{208}$Pb as suggested in
% Ref.\cite{RH09}.

The fusion calculations are also performed for the $^{32}$S + $^{208}$Pb and 
$^{40}$Ca + $^{208}$Pb systems in order to investigate the projectile 
mass-number dependence of the effect of the noncollective excitations. 
We have shown that the effect of the noncollective
excitations becomes stronger as the mass number of the 
projectile nucleus increases.
This result can be considered to justify the conventional 
coupled-channels calculation which neglects the noncollective excitations
for relatively light systems.
However, it also shows that the noncollective excitations should be considered
in the calculation for heavy systems, for example, those relevant to a
synthesis of superheavy elements.

For the $^{32}$S + $^{208}$Pb system, the coupled-channels calculations 
with only the inelastic excitations of the colliding nuclei 
do not account for the experimental data. 
That is, the subbarrier fusion cross sections are significantly 
underestimated for this system and
the experimental barrier distribution is much more smeared than that
obtained by the coupled-channels calculation.
The transfer process
should be taken into account for this system simultaneously with
the noncollective excitations
in order to improve the agreement with the data.

We have also calculated the energy dependence of the Q-value distribution 
for the $^{16}$O + $^{208}$Pb system and found
that the contribution from the noncollective excitations 
becomes more and more
important as the incident energy increases. 
This behaviour is qualitatively consistent with
the experimental Q-value distribution for the same system.
The experiment data also indicate the contribution
from transfer channels in this system\cite{evers2}.
Therefore, it will be an interesting future work to study the contribution from
the transfer channels to the Q-value distribution and compare with the
experimental data in a quantitative way. 

In chapter 6, we investigated quasi-elastic scattering for $^{20}$Ne + $^{90,92}$Zr
systems where the conventional coupled-channels calculation fails
to account for the experimental data.
Although the experimental information on 
the noncollective states, especially deformation
parameter, is available for $^{208}$Pb nucleus, the information on the
noncollective states for $^{90,92}$Zr nuclei has not been sufficiently obtained.
Thus, we employed the random matrix theory to describe the noncollective
excitations in the coupled-channels calculation.

Our calculations show that the
noncollective excitations fill the dip between the
two peaks in the barrier distribution for $^{20}$Ne + $^{92}$Zr system and hence 
the peak structure is smeared.
On the other hand, there is only a minor effect
of the noncollective excitations for $^{20}$Ne + $^{90}$Zr system.
Although our calculation does not improve the agreement of the quasi-elastic
scattering cross sections below the barrier,
we have shown that the magnitude of the noncollective
effect is considerably different between the two systems.
In these calculations, the parameters in the random matrix model, which determine
the transition strength, are taken to be the same
between the two systems.
Therefore, the difference of the noncollective effect arises from the difference
in the level density 
for the low-lying states
which enters into the form factor in the coupling matrix
elements in the random matrix model.

We also calculated the Q-value distribution for $^{20}$Ne + $^{90,92}$Zr
scattering and compared with the experimental data taken at $E_{\rm CM}$ = 51.85
MeV. Our results show that the contribution from the noncollective excitations
is so small in this energy that the calculated Q-value distribution is almost
unchanged. We compared the Q-value distributions for $^{20}$Ne + 
$^{90,92}$Zr systems, and found that both the experimental data 
and the calculation indicate that
the Q-value distributions at this energy do not differ much between the two
systems.
We also calculated the energy dependence of the Q-value distribution around
the Coulomb barrier energy, 
and found that it shows the similar behavior to the $^{16}$O +
$^{208}$Pb system. That is, while at energies below the Coulomb barrier, the
contribution from the elastic and the collective channels is dominant,
the contribution from the
noncollective channels becomes more and more important as the incident energy
increases. For $^{20}$Ne + $^{92}$Zr system, the
noncollective excitations also contribute
to construct a peak structure in
the Q-value distribution at above barrier energies.
Therefore, in order to see the effect of 
the noncollective excitations of Zr
isotopes on the Q-value distribution, 
it will be necessary to measure the data
at above barrier energies and see the energy dependence.

In order to see the effect of the noncollective excitations of $^{92}$Zr on
other reactions, we investigated $^{16}$O + $^{92}$Zr
and $^{28}$Si + $^{92}$Zr systems.
For the $^{16}$O + $^{92}$Zr system,
the noncollective excitations have minor effect
on the barrier distribution. 
For this system, the barrier distribution is not structured
even in the absence of the noncollective excitations.
Thus, the smearing due to the noncollective excitations does not change the
shape of the barrier distribution.
For the $^{28}$Si + $^{92}$Zr system, the noncollective excitations smear the
barrier distribution.
Although it seems to somewhat deteriorate the agreement with the
experimental data, it will be more or less cured by readjusting the potential
parameters, and thus we conclude that our calculations are not inconsistent with
the previously measured data .

We finally investigated the $^{24}$Mg + $^{90,92}$Zr systems
since $^{24}$Mg is a prolately deformed nucleus,
and a similar noncollective
effect to $^{20}$Ne + $^{90,92}$Zr can be expected.
In fact, our calculation indicates a similar smearing effect for $^{24}$Mg +
$^{92}$Zr system, while the barrier distribution for $^{24}$Mg + $^{90}$Zr
system is not significantly changed by the noncollective excitations.
Since the barrier distributions has not been obtained experimentally
for this system, our
calculation gives a prediction. If the experimental barrier distribution
exhibits the smeared structure for $^{24}$Mg + $^{92}$Zr system, the importance
of the noncollective excitations will become robust.

In this thesis,
we have investigated the role of noncollective excitations in heavy-ion
reactions
motivated by the quasi-elastic scattering experiment for $^{20}$Ne +
$^{90,92}$Zr systems.
Since the conventional coupled-channels analyses take into account
only the collective excitations, the effect of the noncollective excitations
had not been clarified in the previous studies.
In order to study the effect of the noncollective excitations on the 
fusion and the quasi-elastic scattering, we investigated several
systems.
We have found that the effect of the noncollective excitations is to
smear the barrier structure which is constructed by the collective excitations.
They also contribute to the Q-value distribution above barrier energies.
However, the noncollective effect on the
fusion cross sections is not large
compared to that of the collective excitations, and the gross structure
of the barrier distribution is still determined by the collective excitations.
Thus, these results can be considered to justify the success of
the conventional coupled-channels analyses in medium-heavy systems.
On the other hand, if one is interested in the
detailed structure of the barrier distribution, our results suggest that the
effect of the noncollective excitations can be important.
In fact, our results show that in some systems,
the effect of the noncollective excitations
is important, e.g. $^{20}$Ne + $^{92}$Zr and $^{24}$Mg + $^{92}$Zr systems,
while in other systems,
e.g. in $^{16}$O + $^{208}$Pb and $^{20}$Ne + $^{90}$Zr systems,
it is less important.
It will be an interesting question to clarify a general criterion 
for a need to take into account the noncollective excitations.
In this respect, we have clarified several conditions.
The first is that the noncollective excitations should be taken into account for
heavy systems such as those relevant to a synthesis of superheavy elements.
This is because the coupling effect becomes effectively strong as the charge
product of projectile and target becomes large.
The second is that the noncollective excitations are important for systems
with large number of levels or large level density at relatively low
excitation energy region. This has been clarified from the comparison of $^{20}$Ne +
$^{90}$Zr and $^{20}$Ne + $^{92}$Zr systems.
And the third point is the properties of coexisting collective excitations which
dominate the barrier structure.
As in the case of $^{16}$O + $^{92}$Zr reaction,
if the collective excitations do not yield a clear
structure in the barrier distribution, the noncollective excitations 
do not significantly alter the barrier distribution.
In order to proceed this study, it will be necessary to investigate systems
where the effect of the noncollective excitations
shows up in a different way, as in the
case of $^{90,92}$Zr targets.

The effect of the noncollective excitations will be important for
deep inelastic collisions(DIC) in massive systems, 
where a large amount of the kinetic energy
is dissipated into the internal excitations.
In the previous analyses of DIC, a classical friction models have been used.
It will be an interesting future study to apply the model presented in
this thesis to DIC and
describe the dissipation phenomena quantum mechanically,
instead of using classical models.
This kind of study will be useful not only in nuclear reactions but also
in other fields, since the effect of the dissipation on the reaction process 
has been studied from a general point of view\cite{CL81,CL83}.

We employed random matrix model to describe the noncollective excitations of
$^{90,92}$Zr because the experimental information on the noncollective
states are limited.
Another possible description of the noncollective excitations will be
to calculate the excited states microscopically and
use the theoretical transition probabilities.
In such calculations, it will be necessary to take into account
the paring effect for $^{92}$Zr nucleus.
The investigations on this direction will lead to the development of the
microscopic description of the heavy-ion reactions.

In our calculations for the $^{20}$Ne + $^{90,92}$Zr systems, 
we have not taken into account transfer reactions because
the experimental total transfer cross sections
have been found to be 
almost the same between the two systems.
However, if one looks at the cross sections for each 
transfer processes separately, they are different between these
systems, which may affect the barrier distribution in a different way.
Therefore, it is a challenging future work to study the effect of 
transfer processes as well as noncollective excitations.
It will be also interesting to investigate the transfer effect on the Q-value
distribution which has been suggested in the experiment for $^{16}$O + $^{208}$Pb
system\cite{evers2}.
The study of the noncollective excitations and the transfer reactions
will contribute to the development of the coupled-channels method
and the further understanding of the reaction processes.


\include{end}
