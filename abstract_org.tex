\documentclass[12pt]{report}
%\usepackage{fancybox}
\usepackage{geometry}
\usepackage{amsmath}
\usepackage{txfonts}
\usepackage{layout}
\usepackage{setspace}
%\usepackage{mathptmx}
\geometry{a4paper, left=22mm, right=22mm, top=25mm, bottom=25mm}
\usepackage{wrapfig}
\usepackage[dvipdfm]{graphicx,hyperref}
\usepackage{mediabb}
\setstretch{1.3}
\title{\Huge\bf{Role of Noncollective Excitations in Low-Energy Heavy-Ion Fusion Reaction
and Quasi-Elastic Scattering}}
\author{\\\\\\\\\\\\\\\\\\\\\\\\\\\\{\it\Large Department of Physics, Faculty of Science, Tohoku University}\\\\
\Huge Shusaku Yusa}
\date{March, 2013}
\begin{document}

\begin{center}
{\Large\bf Abstract}
\end{center}

In heavy-ions reactions, a cancellation between
an attractive nuclear potential and the repulsive 
Coulomb potential makes a potential barrier called the 
Coulomb barrier between the colliding nuclei. 
In heavy-ion reactions around the Coulomb barrier energy,
the coupling between the relative motion and internal excitations 
of the colliding nuclei has been found to play an important role.
It has been well known that subbarrier fusion cross sections are
significantly enhanced due to the coupling effect, 
compared to a prediction of a simple potential model.

In order to take into account the coupling effect, a coupled-channels method
has been employed as a standard approach.
Conventionally, only a few low-lying collective excitations such as vibrational 
excitation or rotational excitations in deformed nuclei have been taken into
account. The coupled-channels method has successfully accounted for experimental
data for heavy-ion fusion reactions as well as quasi-elastic scattering.

Recently, however, a few experimental data which cannot be accounted for
by the conventional coupled-channels method have been obtained.
These include the quasi-elastic scattering experiment for 
$^{20}$Ne + $^{90,92}$Zr systems and the fusion and quasi-elastic scattering
experiments for $^{16}$O + $^{208}$Pb system.
The conventional coupled-channels calculations, which take into
account only the collective excitations of the colliding nuclei, failed to
reproduce the data, and the noncollective excitations, which are not included
in the usual coupled-channels calculations, 
are suggested to play an important role in these systems.
The noncollective excitations have not been taken into account 
explicitly in previous studies of the low-energy heavy-ion reactions,
and their role has not been clarified.
In this thesis, we explicitly
take into account the noncollective excitations
in the coupled-channels calculations and
clarify their role
in low-energy heavy-ion reactions.

At first, the fundamental properties of the collective and the 
noncollective excited states are reviewed.
By using the liquid drop model, we discuss how the regularity of the
collective excited states appears. We also mention
an interpretation of the collective and the
noncollective excited states from a microscopic point of view.

The theoretical frame work for the study of the low-energy
heavy-ion reactions is discussed in the next.
The coupled-channels formalism is reviewed and the barrier distribution
method is introduced. We discuss the effect
of the collective excitations on heavy-ion fusion reactions
through the calculation
of the fusion barrier distribution.
We also review the random matrix theory and its applications, as
we employ the model
of Weidenm\"uller {\it et al.} for deep inelastic collisions
based on the random matrix
theory for the description of the noncollective excitations.

We start our investigation of the role of the noncollective
excitations with 
$^{16}$O + $^{208}$Pb system[2].
For this system, the energy dependence of the Q-value distribution
(a distribution of the energy of a scattered particle)
has been experimentally obtained. 
The experimental data show that the contribution from the
higher excitation energy region,
which can be considered as the noncollective excitations,
increases as the incident energy
increases.
For $^{208}$Pb, the information on
the excited states up to a rather high excitation energy
has been obtained 
by the high precision proton inelastic scattering experiments.
We describe the noncollective excitations of $^{208}$Pb using
this information as inputs of calculations.
We show that the energy dependence of the calculated 
Q-value distribution
is consistent with the experimental data.

We then study the role of the noncollective excitations in the
quasi-elastic scattering for $^{20}$Ne + $^{90,92}$Zr systems.
For these systems, the experimental quasi-elastic barrier distributions
show different behavior between the two systems,
that is, the barrier distribution
for the $^{20}$Ne + $^{92}$Zr system is much more smeared than that 
of the $^{20}$Ne + $^{90}$Zr system.
However, the coupled-channels calculation cannot yield smeared barrier 
distribution for $^{20}$Ne + $^{92}$Zr system, and thus cannot
account for the difference
in the barrier distribution if only the collective excitations are taken
into account.
In order to see whether the noncollective excitations cure this problem,
we take into account the noncollective excitations of Zr
isotopes in the calculation.
For the description of the noncollective excitations,
we use the random matrix theory, because the information on
the excited states has not been sufficiently obtained for Zr isotopes
in contrast to the case of $^{208}$Pb.
We show that, by taking into account the noncollective excitations,
the quasi-elastic barrier distribution for the $^{20}$Ne + $^{92}$Zr
system is significantly altered, while for the $^{20}$Ne + $^{90}$Zr system,
the effect of the noncollective excitations is found to be small.
Although our calculation does not improve the agreement of the quasi-elastic
scattering cross sections below the barrier,
we show that the magnitude of the noncollective
effect is considerably different between the two systems.
This difference originates from the level density of the Zr isotopes.
That is, since $^{90}$Zr is a closed shell nucleus with 50 neutrons and
$^{92}$Zr has two extra neutrons, a large number of the noncollective excited
states appear in $^{92}$Zr nucleus.
We also show that our calculation predicts a similar
effect of the noncollective excitations for
$^{24}$Mg + $^{90,92}$Zr systems.\\


[1] S. Yusa, K. Hagino, N. Rowley, 
\href{http://dx.doi.org/10.1103/PhysRevC.82.024606}
{Phys. Rev. C {\bf 82}, 024606 (2010)}.

[2] S. Yusa, K. Hagino, N. Rowley, 
\href{http://dx.doi.org/10.1103/PhysRevC.85.054601}
{Phys. Rev. C {\bf 85}, 054601 (2012)}.
